\documentclass[12pt,reqno]{amsart}


\newcommand\hmmax{0}
\newcommand\bmmax{0}
\usepackage{graphicx}
\usepackage{caption}
\usepackage{times}
\usepackage[colorlinks=true,linkcolor=blue,citecolor=blue]{hyperref}%
\usepackage{comment}

\usepackage{xcolor}
%\usepackage{hyperref}
%\hypersetup{
   %linktoc=all,
    %colorlinks=true
    %citecolor=blue
%}
\usepackage{stmaryrd}
\usepackage{dsfont}

\newtheorem{theorem}{Theorem}[section]
\newtheorem{lemma}[theorem]{Lemma}
\newtheorem{corollary}[theorem]{Corollary}
\newtheorem{prop}[theorem]{Proposition}
\newtheorem{ass}[theorem]{Assumption}
\newtheorem{notation}[theorem]{Notation}
\theoremstyle{definition}
\newtheorem{definition}[theorem]{Definition}
\newtheorem{construction}[theorem]{Construction}
\theoremstyle{remark}
\newtheorem{remark}[theorem]{Remark}
\newtheorem*{rem}{Remark}
\numberwithin{equation}{section}

\urlstyle{same}

\usepackage[top=1.3in, bottom=1.3in, left=1.3in, right=1.3in]{geometry}
\usepackage[scr]{rsfso}
\usepackage[english]{babel}
\usepackage{fancyhdr}
\usepackage{amsmath}
\usepackage{amsthm}
\usepackage{amssymb}
\usepackage{eucal}
\usepackage{enumitem}
\setlist{leftmargin=*}
\usepackage[integrals]{wasysym}


\makeatletter
\newsavebox{\@brx}
\newcommand{\llangle}[1][]{\savebox{\@brx}{\(\m@th{#1\langle}\)}%
  \mathopen{\copy\@brx\kern-0.5\wd\@brx\usebox{\@brx}}}
\newcommand{\rrangle}[1][]{\savebox{\@brx}{\(\m@th{#1\rangle}\)}%
  \mathclose{\copy\@brx\kern-0.5\wd\@brx\usebox{\@brx}}}
\makeatother


\newcommand\nc{\newcommand}
\nc{\E}{\mathbb{E}}
\nc{\R}{\mathbb R}
\nc{\C}{\mathbb C}
\nc{\Z}{\mathbb Z}
\nc{\wt}{\widetilde}
\nc{\rnc}{\renewcommand}
\nc{\e}{\varepsilon}
\nc{\grad}{\nabla}
%\nc{\fsp}{\fontdimen2\font=2.21pt}
\nc{\fsp}{}

\rnc{\t}{{t}}
\nc{\s}{{s}}
\nc{\x}{{x}}
\nc{\y}{{y}}
\nc{\w}{{w}}
\nc{\z}{{z}}
\rnc{\r}{{r}}
\rnc{\k}{{k}}
\rnc{\j}{{j}}
\nc{\m}{{m}}
\nc{\n}{{n}}
\rnc{\i}{{i}}
\nc{\p}{{p}}
\rnc{\textstyle}{{}}



%\rnc{\t}{\mathrm{t}}
%\nc{\s}{\mathrm{s}}
%\nc{\x}{\mathrm{x}}
%\nc{\y}{\mathrm{y}}
%\nc{\w}{\mathrm{w}}
%\nc{\z}{\mathrm{z}}
%\rnc{\r}{\mathrm{r}}
%\rnc{\k}{\mathrm{k}}
%\rnc{\j}{\mathrm{j}}
%\nc{\m}{\mathrm{m}}
%\nc{\n}{\mathrm{n}}
%\rnc{\i}{\mathrm{i}}
%\nc{\p}{\mathrm{p}}


\nc{\abbr}[1]{{\sc\lowercase{#1}}}

\rnc{\leq}{\leqslant}
\rnc{\geq}{\geqslant}
\rnc{\d}{\mathrm{d}}
\newenvironment{nouppercase}{%
  \let\uppercase\relax%
  \renewcommand{\uppercasenonmath}[1]{}}{}
\pagestyle{plain}




\title{\Large Math 154: Probability Theory, HW 8\vspace{-0.1cm}}





\usepackage{setspace}
\begin{document}
%\pdfrender{StrokeColor=gray,TextRenderingMode=2,LineWidth=0.01pt}
\setstretch{0.97}
\begin{nouppercase}
\maketitle
\end{nouppercase}
%%%
\section*{Due April 2, 2024 by 9am}
%%%
\emph{Remember, if you are stuck, take a look at the lemmas/theorems/examples from class, and see if anything looks familiar.}
%%%
\section{Some practice with Markov chains}
%%%
%%%
\subsection{Classification of states}\label{subsection:classification}
%%%
Consider the state space $\{A,B,C,D\}$. For each Markov chain below (specified by its transition matrix), specify which states (i.e. which of $A,B,C,D$) are recurrent and which are transient. (Recall a transition matrix $P$ has entries given by $P_{ij}=\mathbb{P}[i\to j]$.)
%%%
\begin{enumerate}
\item $P_{1}=\begin{pmatrix}\frac12&\frac12&0&0\\\frac12&\frac12&0&0\\\frac14&0&\frac14&\frac12\\0&0&0&1\end{pmatrix}$\\
\item $P_{2}=\begin{pmatrix}0&\frac12&\frac12&0\\\frac13&0&0&\frac23\\1&0&0&0\\0&0&1&0\end{pmatrix}$
\item Find two row vectors $\pi_{1}$ and $\pi_{2}$ of length $4$ such that $\pi_{1}P_{1}=\pi_{1}$ and $\pi_{2}P_{1}=\pi_{2}$. Your two row vectors cannot be scalar multiplies of each (e.g. they must be linearly independent). What do you notice about the sign of each entry in $\pi_{1},\pi_{2}$?
\end{enumerate}
%%%
%%%
\subsection{A nice trick in computing long-time behavior of a Markov chain}
%%%
Consider $P_{1}$ from Problem \ref{subsection:classification}. We will see that diagonalization from linear algebra is actually useful.
%%%
\begin{enumerate}
\item Compute $\mathrm{Tr} P_{1}$ and $\det P_{1}$.
\item Compute the eigenvalues of $P_{1}$. (\emph{Hint}: the eigenvalues sum to the trace, and they multiply to the determinant. Use part (3) in Problem \ref{subsection:classification}.) 
\item Label the eigenvalues as $\lambda_{1}\geq\lambda_{2}\geq\lambda_{3}\geq\lambda_{4}$, and let $\mathbf{v}_{1},\mathbf{v}_{2},\mathbf{v}_{3},\mathbf{v}_{4}$ be the associated left eigenvectors, so that $\mathbf{v}_{i}P_{1}=\lambda_{i}\mathbf{v}_{i}$. Show that $|\lambda_{3}|,|\lambda_{4}|<1$. Deduce that for $i=3,4$, we have $\mathbf{v}_{i}P_{1}^{n}\to\vec{0}$ as $n\to\infty$, where $\vec{0}=(0,0,0,0)$.
\item Any vector $\mathbf{v}$ can be written as a linear combination $\mathbf{v}=\alpha_{1}\mathbf{v}_{1}+\alpha_{2}\mathbf{v}_{2}+\alpha_{3}\mathbf{v}_{3}+\alpha_{4}\mathbf{v}_{4}$. Show that $\mathbf{v}P_{1}^{n}\to\alpha_{1}\mathbf{v}_{1}+\alpha_{2}\mathbf{v}_{2}$ as $n\to\infty$. This shows that the long-time behavior of the $P_{1}$ Markov chain is rather simple! 
\end{enumerate}
%%%
%%%
\subsection{Random walk in dimension $2$}
%%%
Let $\mathbf{X}(n)=(X_{1}(n),X_{2}(n))$, where $X_{1},X_{2}$ are independent symmetric simple random walks such that $X_{1}(0),X_{2}(0)=0$ and $n\geq0$ is an integer. 
%%%
\begin{enumerate}
\item Show that for any $n\geq0$, we have 
%
\begin{align*}
\mathbb{P}[\mathbf{X}(2n)=(0,0)]=\binom{2n}{n}^{2}2^{-4n}.
\end{align*}
%
Deduce that $\mathbb{P}[\mathbf{X}(2n)=(0,0)]\geq Cn^{-1}$ for all $n\geq1$, where $C\geq0$ is some fixed constant. (\emph{Hint}: use independence of $X_{1},X_{2}$.)
\item Show that 
%
\begin{align*}
\sum_{n=1}^{\infty}\mathbb{P}[\mathbf{X}(n)=(0,0)]=\infty.
\end{align*}
%
\end{enumerate}
%%%
%%%
\subsection{Random walk in dimensions greater than or equal to $3$}
%%%
Let $\mathbf{X}(n)=(X_{1}(n),\ldots,X_{d}(n))$, where $X_{1},\ldots,X_{d}$ are independent symmetric simple random walks such that $X_{1}(0),\ldots,X_{d}(0)=0$, and $n\geq0$ is an integer and $d\geq3$ is fixed.
%%%
\begin{enumerate}
\item Show that $\mathbb{P}[\mathbf{X}(2n)=(0,\ldots,0)]\leq Cn^{-d/2}$ for all $n\geq1$, where $C$ depends only on $d$.
\item Show that $\mathbf{X}$ if $d\geq3$, then 
%
\begin{align*}
\sum_{n=1}^{\infty}\mathbb{P}[\mathbf{X}(n)=(0,\ldots,0)]<\infty.
\end{align*}
%
\end{enumerate}
%%%
\subsection{Asymmetric simple random walk in dimension $1$}
%%%
Suppose $\{X(n)\}_{n\geq1}$ is a sequence of i.i.d. random variables such that 
%
\begin{align*}
\mathbb{P}[X(n+1)=x]=\begin{cases}p&x=1\\1-p&x=-1\\0&\mathrm{else}\end{cases}
\end{align*}
%
where $p\neq\frac12$. Define $S(n)=X(1)+\ldots+X(n)$ to be the random walk with $S(0)=0$.
%%%
\begin{enumerate}
\item Show that the process $M_{n}=S(n)-(2p-1)n$ is a martingale with respect to the sequence $\{X(k)\}_{k\geq1}$. Show that $|M_{n+1}-M_{n}|\leq 10$ for all $n\geq0$.
\item Show that for some constant $C>0$ independent of $n\geq0$, we have
%
\begin{align*}
\mathbb{P}\left[|M_{n}|\geq n^{2/3}\right]\leq \exp\left\{-Cn^{1/3}\right\}
\end{align*}
%
\item Show that $\mathbb{P}[S(n)=0]\leq\mathbb{P}[|M_{n}|\geq n^{2/3}]$ for $n$ large enough. Using the bound $\exp\{-Cn^{1/3}\}\leq C_{2}n^{-2}$ for some $C_{2}>0$ fixed, deduce that $X$ has $0$ as a transient state. (\emph{Hint}: the assumption $p\neq\frac12$ is crucial.)
\end{enumerate}
%%%
\end{document}
%%%