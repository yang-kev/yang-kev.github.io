\documentclass[12pt,reqno]{amsart}


\newcommand\hmmax{0}
\newcommand\bmmax{0}
\usepackage{graphicx}
\usepackage{caption}
\usepackage{times}
\usepackage[colorlinks=true,linkcolor=blue,citecolor=blue]{hyperref}%
\usepackage{comment}

\usepackage{xcolor}
%\usepackage{hyperref}
%\hypersetup{
   %linktoc=all,
    %colorlinks=true
    %citecolor=blue
%}
\usepackage{stmaryrd}
\usepackage{dsfont}

\newtheorem{theorem}{Theorem}[section]
\newtheorem{lemma}[theorem]{Lemma}
\newtheorem{corollary}[theorem]{Corollary}
\newtheorem{prop}[theorem]{Proposition}
\newtheorem{ass}[theorem]{Assumption}
\newtheorem{notation}[theorem]{Notation}
\theoremstyle{definition}
\newtheorem{definition}[theorem]{Definition}
\newtheorem{construction}[theorem]{Construction}
\theoremstyle{remark}
\newtheorem{remark}[theorem]{Remark}
\newtheorem*{rem}{Remark}
\numberwithin{equation}{section}

\urlstyle{same}

\usepackage[top=1.3in, bottom=1.3in, left=1.3in, right=1.3in]{geometry}
\usepackage[scr]{rsfso}
\usepackage[english]{babel}
\usepackage{fancyhdr}
\usepackage{amsmath}
\usepackage{amsthm}
\usepackage{amssymb}
\usepackage{eucal}
\usepackage{enumitem}
\setlist{leftmargin=*}
\usepackage[integrals]{wasysym}


\makeatletter
\newsavebox{\@brx}
\newcommand{\llangle}[1][]{\savebox{\@brx}{\(\m@th{#1\langle}\)}%
  \mathopen{\copy\@brx\kern-0.5\wd\@brx\usebox{\@brx}}}
\newcommand{\rrangle}[1][]{\savebox{\@brx}{\(\m@th{#1\rangle}\)}%
  \mathclose{\copy\@brx\kern-0.5\wd\@brx\usebox{\@brx}}}
\makeatother


\newcommand\nc{\newcommand}
\nc{\E}{\mathbb{E}}
\nc{\R}{\mathbb R}
\nc{\C}{\mathbb C}
\nc{\Z}{\mathbb Z}
\nc{\wt}{\widetilde}
\nc{\rnc}{\renewcommand}
\nc{\e}{\varepsilon}
\nc{\grad}{\nabla}
%\nc{\fsp}{\fontdimen2\font=2.21pt}
\nc{\fsp}{}

\rnc{\t}{{t}}
\nc{\s}{{s}}
\nc{\x}{{x}}
\nc{\y}{{y}}
\nc{\w}{{w}}
\nc{\z}{{z}}
\rnc{\r}{{r}}
\rnc{\k}{{k}}
\rnc{\j}{{j}}
\nc{\m}{{m}}
\nc{\n}{{n}}
\rnc{\i}{{i}}
\nc{\p}{{p}}
\rnc{\textstyle}{{}}



%\rnc{\t}{\mathrm{t}}
%\nc{\s}{\mathrm{s}}
%\nc{\x}{\mathrm{x}}
%\nc{\y}{\mathrm{y}}
%\nc{\w}{\mathrm{w}}
%\nc{\z}{\mathrm{z}}
%\rnc{\r}{\mathrm{r}}
%\rnc{\k}{\mathrm{k}}
%\rnc{\j}{\mathrm{j}}
%\nc{\m}{\mathrm{m}}
%\nc{\n}{\mathrm{n}}
%\rnc{\i}{\mathrm{i}}
%\nc{\p}{\mathrm{p}}


\nc{\abbr}[1]{{\sc\lowercase{#1}}}

\rnc{\leq}{\leqslant}
\rnc{\geq}{\geqslant}
\rnc{\d}{\mathrm{d}}
\newenvironment{nouppercase}{%
  \let\uppercase\relax%
  \renewcommand{\uppercasenonmath}[1]{}}{}
\pagestyle{plain}




\title{\Large Math 154: Probability Theory, HW 9\vspace{-0.1cm}}





\usepackage{setspace}
\begin{document}
%\pdfrender{StrokeColor=gray,TextRenderingMode=2,LineWidth=0.01pt}
\setstretch{0.97}
\begin{nouppercase}
\maketitle
\end{nouppercase}
%%%
\section*{Due April 16, 2024 by 9am}
%%%
\emph{Remember, if you are stuck, take a look at the lemmas/theorems/examples from class, and see if anything looks familiar.}
%%%
\section{Getting our hands on Brownian motion}
%%%
%%%
\subsection{A computation}
%%%
Consider the integral $\int_{0}^{t}\mathbf{B}_{s}^{2}\d s$.
%%%
\begin{enumerate}
\item Compute $\E\int_{0}^{t}\mathbf{B}_{s}^{2}\d s$.
\item Compute $\E|\int_{0}^{t}\mathbf{B}_{s}^{2}\d s|^{2}$. (\emph{Hint}: as in class, square the integral to get a double integral over $0\leq r\leq s\leq t$. For $r\leq s$, it may then help to write $\mathbf{B}_{s}^{2}\mathbf{B}_{r}^{2}=(\mathbf{B}_{s}-\mathbf{B}_{r}+\mathbf{B}_{r})^{2}\mathbf{B}_{r}^{2}=(\mathbf{B}_{s}-\mathbf{B}_{r})^{2}\mathbf{B}_{r}^{2}+2(\mathbf{B}_{s}-\mathbf{B}_{r})\mathbf{B}_{r}^{3}+\mathbf{B}_{r}^{4}$. Now use independence of increments and knowledge of the distribution of increments.)
\item Deduce the variance of $\int_{0}^{t}\mathbf{B}_{s}^{2}\d s$.
\end{enumerate}
%%%
%%%
\subsection{Brownian Gambler's ruin (\emph{Hint}: use optional stopping!)}
%%%
Let $\mathbf{B}$ be Brownian motion, and fix $a,b>0$. Let $\tau_{a,b}$ be the first time $\tau$ such that $\mathbf{B}_{\tau}\in\{-a,b\}$.
%%%
\begin{enumerate}
\item Find the probability that $\mathbf{B}_{\tau_{a,b}}=a$.
\item Compute $\E\tau_{a,b}$.
\end{enumerate}
%%%
%%%
\subsection{Moment generating function of Gaussians, Brownian motion style}
%%%
Consider the process $\mathbf{M}_{t}:=\exp\left\{\lambda\mathbf{B}_{t}-\mu t\right\}$, where $\lambda,\mu\in\R$.
%%%
\begin{enumerate}
\item Fix $\lambda\in\R$. For which $\mu=\mu(\lambda)\in\R$ does $\mathbf{M}$ satisfy the martingale property? ($\mu(\lambda)$ will depend on $\lambda$)
\item Fix $\lambda\in\R$. Show that $\E\mathbf{M}_{1}=1$. In what follows, we will always take $\mathbf{M}_{t}$ for this choice of $\mu=\mu(\lambda)$.
\item Deduce that if $Z\sim N(0,1)$, then $\E e^{\lambda Z}=e^{\lambda^{2}/2}$. (\emph{Hint}: recall $\mathbf{B}_{1}\sim N(0,1)$.)
\end{enumerate}
%%%
%%%
\subsection{Ergodicity of the OU process}
%%%
Suppose $X_{t}$ is an OU process with initial condition $X_{0}$, that is $\d X_{t}=- X_{t}\d t+\d\mathbf{B}_{t}$, where $\mathbf{B}_{t}$ is a Brownian motion.
%%%
\begin{enumerate}
\item Show that $N(0,1)$ is an invariant distribution for the OU process (see the notes for what this means).
\item Let $Z_{t}$ be an OU process with initial condition $Z_{0}\sim N(0,1)$. That is, $\d Z_{t}=- Z_{t}+\d\mathbf{B}_{t}$, where $\mathbf{B}$ is the \emph{same} Brownian motion from above. Define $Y_{t}=X_{t}-Z_{t}$. Show that $Y_{t}=Y_{0}e^{-t}$ for all $t\geq0$. Deduce that $Y_{t}\to0$ as $t\to\infty$. (\emph{Hint}: compute the differential equation solved by $Y_{t}$ using the SDEs for $X_{t},Z_{t}$; you can use that any solution to $f'(t)=-f(t)$ is given by $f(t)=f(0)e^{-t}$.)
\end{enumerate}
%%%
%%%
\subsection{Brownian bridge}
%%%
The Brownian bridge is a ``Brownian motion conditioned to hit $0$ at time $1$". The point of this exercise is to make this precise in a more natural way. 

Let $\{z_{k}\}_{k=1}^{\infty}$ be a collection of i.i.d. $N(0,1)$ random variables. For any $N>0$, define 
%
\begin{align*}
\mathbf{Z}^{(N)}_{t}:=\sum_{k=1}^{N}\frac{z_{k}\sqrt{2}}{k\pi}\sin(k\pi t).
\end{align*}
%
Show that $\mathbf{Z}^{(N)}_{0}=\mathbf{Z}^{(N)}_{1}=0$. Show that $\E\mathbf{Z}^{(N)}_{t}=0$ and that
%
\begin{align*}
\E|\mathbf{Z}^{(N)}_{t}-\mathbf{Z}^{(M)}_{t}|^{2}\to_{N,M\to\infty}0.
\end{align*}
%
%%%












\end{document}
%%%