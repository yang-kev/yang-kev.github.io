\documentclass[12pt,reqno]{amsart}


\newcommand\hmmax{0}
\newcommand\bmmax{0}
\usepackage{graphicx}
\usepackage{caption}
\usepackage{times}
\usepackage[colorlinks=true,linkcolor=blue,citecolor=blue]{hyperref}%

\usepackage{xcolor}
%\usepackage{hyperref}
%\hypersetup{
   %linktoc=all,
    %colorlinks=true
    %citecolor=blue
%}
\usepackage{stmaryrd}
\usepackage{dsfont}

\newtheorem{theorem}{Theorem}[section]
\newtheorem{lemma}[theorem]{Lemma}
\newtheorem{corollary}[theorem]{Corollary}
\newtheorem{prop}[theorem]{Proposition}
\newtheorem{ass}[theorem]{Assumption}
\newtheorem{notation}[theorem]{Notation}
\theoremstyle{definition}
\newtheorem{definition}[theorem]{Definition}
\newtheorem{construction}[theorem]{Construction}
\theoremstyle{remark}
\newtheorem{remark}[theorem]{Remark}
\newtheorem*{rem}{Remark}
\numberwithin{equation}{section}

\urlstyle{same}

\usepackage[top=1.3in, bottom=1.3in, left=1.3in, right=1.3in]{geometry}
\usepackage[scr]{rsfso}
\usepackage[english]{babel}
\usepackage{fancyhdr}
\usepackage{amsmath}
\usepackage{amsthm}
\usepackage{amssymb}
\usepackage{eucal}
\usepackage{enumitem}
\setlist{leftmargin=*}
\usepackage[integrals]{wasysym}


\makeatletter
\newsavebox{\@brx}
\newcommand{\llangle}[1][]{\savebox{\@brx}{\(\m@th{#1\langle}\)}%
  \mathopen{\copy\@brx\kern-0.5\wd\@brx\usebox{\@brx}}}
\newcommand{\rrangle}[1][]{\savebox{\@brx}{\(\m@th{#1\rangle}\)}%
  \mathclose{\copy\@brx\kern-0.5\wd\@brx\usebox{\@brx}}}
\makeatother


\newcommand\nc{\newcommand}
\nc{\E}{\mathbb{E}}
\nc{\R}{\mathbb R}
\nc{\C}{\mathbb C}
\nc{\Z}{\mathbb Z}
\nc{\wt}{\widetilde}
\nc{\rnc}{\renewcommand}
\nc{\e}{\varepsilon}
\nc{\grad}{\nabla}
%\nc{\fsp}{\fontdimen2\font=2.21pt}
\nc{\fsp}{}

\rnc{\t}{{t}}
\nc{\s}{{s}}
\nc{\x}{{x}}
\nc{\y}{{y}}
\nc{\w}{{w}}
\nc{\z}{{z}}
\rnc{\r}{{r}}
\rnc{\k}{{k}}
\rnc{\j}{{j}}
\nc{\m}{{m}}
\nc{\n}{{n}}
\rnc{\i}{{i}}
\nc{\p}{{p}}
\rnc{\textstyle}{{}}



%\rnc{\t}{\mathrm{t}}
%\nc{\s}{\mathrm{s}}
%\nc{\x}{\mathrm{x}}
%\nc{\y}{\mathrm{y}}
%\nc{\w}{\mathrm{w}}
%\nc{\z}{\mathrm{z}}
%\rnc{\r}{\mathrm{r}}
%\rnc{\k}{\mathrm{k}}
%\rnc{\j}{\mathrm{j}}
%\nc{\m}{\mathrm{m}}
%\nc{\n}{\mathrm{n}}
%\rnc{\i}{\mathrm{i}}
%\nc{\p}{\mathrm{p}}


\nc{\abbr}[1]{{\sc\lowercase{#1}}}

\rnc{\leq}{\leqslant}
\rnc{\geq}{\geqslant}
\rnc{\d}{\mathrm{d}}
\newenvironment{nouppercase}{%
  \let\uppercase\relax%
  \renewcommand{\uppercasenonmath}[1]{}}{}
\pagestyle{plain}




\title{\Large Math 154: Probability Theory, HW 1\vspace{-0.1cm}}





\usepackage{setspace}
\begin{document}
%\pdfrender{StrokeColor=gray,TextRenderingMode=2,LineWidth=0.01pt}
\setstretch{0.97}
\begin{nouppercase}
\maketitle
\end{nouppercase}
%%%
\section*{Due January 30, 2024 by 9am}
%%%
\emph{Remember, if you are stuck, take a look at the lemmas/theorems/examples from class, and see if anything looks familiar.}
%%%
\section{Some practice}
%%%
%%%
\subsection{Dice}
%%%
A traditional fair dice (with numbers $1$ through $6$) is thrown twice. Assuming that the rolls are independent, compute the probability that:
%%%
\begin{enumerate}
\item exactly one $6$ is thrown
\item both rolls are odd numbers
\item the sum of the rolls is $4$
\item the sum of the rolls is divisible by $3$
\end{enumerate}
%%%
%%%
\subsection{Coins}
%%%
Take a coin where the probability of heads is $p$ and the probability of tails is $1-p$. Toss this coin repeatedly. As a function of $n\geq0$ and $p$, compute the probability that on the $n$-th throw:
%%%
\begin{enumerate}
\item heads appears for the first time
\item the number of heads and the number of tails to date are equal
\item exactly two heads have appeared in total to date
\item at least two heads have appeared to date
\end{enumerate}
%%%
%%%
\subsection{Manipulation of events}
%%%
Take two events $A,B$. Show that the probability that exactly one of $A$ or $B$ occurs is equal to $\mathbb{P}(A)+\mathbb{P}(B)-2\mathbb{P}(A\cap B)$.
%%%
\section{Some lemmas}
%%%
%%%
\subsection{Some practice with conditional probability}
%%%
Take three events $A,B,C$. 
%%%
\begin{enumerate}
\item Show that $\mathbb{P}(A\cup B\cup C)=1-\mathbb{P}(A^{c}|B^{c}\cap C^{c})\mathbb{P}(B^{c}|C^{c})\mathbb{P}(C^{c})$.
\item Assume that $A$ and $B$ are independent after conditioning on $C$ (i.e. $\mathbb{P}(A\cap B|C)=\mathbb{P}(A|C)\mathbb{P}(B|C)$). Does this imply that $A$ and $B$ are independent? (If yes, show it. If not, give an example.)
\item Conversely, assume that $A$ and $B$ are independent. Is it true that $A$ and $B$ are independent after conditioning on $C$?
\end{enumerate}
%%%
%%%
\subsection{Bayes' formula}
%%%
\emph{This is one of the most important things in statistics!} Let $\Omega$ be a probability space, and let $A_{1},\ldots,A_{n}$ be a partition of $\Omega$, i.e. $A_{i}\cap A_{j}=\emptyset$ for any $i\neq j$, and $\cup_{i=1}^{n}A_{i}=\Omega$.  Suppose that $\mathbb{P}(A_{i})>0$ for all $i=1,\ldots,n$. Prove that 
%
\begin{align}
\mathbb{P}(A_{j}|B)=\frac{\mathbb{P}(B|A_{j})\mathbb{P}(A_{j})}{\sum_{i=1}^{n}\mathbb{P}(B|A_{i})\mathbb{P}(A_{i})}.\nonumber
\end{align}
%
%
%
%
%%%
\section{Some problems}
%%%
%%%
\subsection{It is very important that $p$ is prime here}
%%%
Set $\Omega=\{1,\ldots,p\}$, where $p$ is prime. Let $\mathscr{F}$ be the set of all subsets of $\Omega$, and $\mathbb{P}(A)=|A|/p$ for any $A\in\mathscr{F}$. Assume that $A$ and $B$ are independent. Show that at least one of $A$ or $B$ is either $\Omega$ or $\emptyset$.
%%%
\subsection{More dice}
%%%
Suppose we throw $N$ independent standard dice, where $N$ is a random number such that $\mathbb{P}(N=i)=2^{-i}$ for all integers $i\geq1$. Let $S$ be the sum of the $N$ values that we throw. Find the probability that:
%%%
\begin{enumerate}
\item $N=2$ conditioning on $S=4$
\item $S=4$ conditioning on $N$ being even
\item $N=2$ conditioning on $S=4$ and the first throw being $1$
\end{enumerate}
%%%
%%%
\subsection{Be careful when driving in the snow}
%%%
There are two roads from $A$ to $B$ and two roads from $B$ to $C$. Each of the four roads is blocked by snow with probability $p$ independently of the others. Find the probability that there is an open road from $A$ to $B$ given that there is no open route from $A$ to $B$ to $C$.
%%%
\subsection{More coins}
%%%
Take a coin where the probability of heads is $p$ and the probability of tails is $1-p$. Let $p_{n}$ be the probability that an even number of heads have been tossed after $n$ tosses. (Zero counts as an even number.) Show that $p_{0}=1$, and show that for any $n\geq1$, 
%
\begin{align*}
p_{n}=p(1-p_{n-1})+(1-p)p_{n-1}.
\end{align*}
%
Using this formula, find a formula in terms of $p$ and $n$ of $p_{n}$. (You do not need to prove that your formula is correct, but you should feel free to.)
%
%
%
%%%
\section{An optional problem (for those who want something to think about)}
%%%
%%%
\subsection{A very inefficient way to board passengers}
%%%
Suppose there are $n$ passengers for a plane with $n$ seats to go from Boston to San Francisco. Each passenger is given their seat number, but the first passenger, say Kevin, to board lost their number and sits in a random seat. The other passengers board one at a time, trying to sit in their seat, or, if their seat is taken, sit in a random available seat. What is the probability that the last passenger sits in their assigned seat? Why?
%%%
\section{A very important question}
%%%
%%%
\subsection{Thank you for answering this!}
%%%
How long did this homework take you? Did you find it hard, easy, something else? (Please feel free to be honest! It's better that I have a sense for if the homework is too long or short or something else.)

















\end{document}
%%%