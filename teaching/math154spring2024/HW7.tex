\documentclass[12pt,reqno]{amsart}


\newcommand\hmmax{0}
\newcommand\bmmax{0}
\usepackage{graphicx}
\usepackage{caption}
\usepackage{times}
\usepackage[colorlinks=true,linkcolor=blue,citecolor=blue]{hyperref}%
\usepackage{comment}

\usepackage{xcolor}
%\usepackage{hyperref}
%\hypersetup{
   %linktoc=all,
    %colorlinks=true
    %citecolor=blue
%}
\usepackage{stmaryrd}
\usepackage{dsfont}

\newtheorem{theorem}{Theorem}[section]
\newtheorem{lemma}[theorem]{Lemma}
\newtheorem{corollary}[theorem]{Corollary}
\newtheorem{prop}[theorem]{Proposition}
\newtheorem{ass}[theorem]{Assumption}
\newtheorem{notation}[theorem]{Notation}
\theoremstyle{definition}
\newtheorem{definition}[theorem]{Definition}
\newtheorem{construction}[theorem]{Construction}
\theoremstyle{remark}
\newtheorem{remark}[theorem]{Remark}
\newtheorem*{rem}{Remark}
\numberwithin{equation}{section}

\urlstyle{same}

\usepackage[top=1.3in, bottom=1.3in, left=1.3in, right=1.3in]{geometry}
\usepackage[scr]{rsfso}
\usepackage[english]{babel}
\usepackage{fancyhdr}
\usepackage{amsmath}
\usepackage{amsthm}
\usepackage{amssymb}
\usepackage{eucal}
\usepackage{enumitem}
\setlist{leftmargin=*}
\usepackage[integrals]{wasysym}


\makeatletter
\newsavebox{\@brx}
\newcommand{\llangle}[1][]{\savebox{\@brx}{\(\m@th{#1\langle}\)}%
  \mathopen{\copy\@brx\kern-0.5\wd\@brx\usebox{\@brx}}}
\newcommand{\rrangle}[1][]{\savebox{\@brx}{\(\m@th{#1\rangle}\)}%
  \mathclose{\copy\@brx\kern-0.5\wd\@brx\usebox{\@brx}}}
\makeatother


\newcommand\nc{\newcommand}
\nc{\E}{\mathbb{E}}
\nc{\R}{\mathbb R}
\nc{\C}{\mathbb C}
\nc{\Z}{\mathbb Z}
\nc{\wt}{\widetilde}
\nc{\rnc}{\renewcommand}
\nc{\e}{\varepsilon}
\nc{\grad}{\nabla}
%\nc{\fsp}{\fontdimen2\font=2.21pt}
\nc{\fsp}{}

\rnc{\t}{{t}}
\nc{\s}{{s}}
\nc{\x}{{x}}
\nc{\y}{{y}}
\nc{\w}{{w}}
\nc{\z}{{z}}
\rnc{\r}{{r}}
\rnc{\k}{{k}}
\rnc{\j}{{j}}
\nc{\m}{{m}}
\nc{\n}{{n}}
\rnc{\i}{{i}}
\nc{\p}{{p}}
\rnc{\textstyle}{{}}



%\rnc{\t}{\mathrm{t}}
%\nc{\s}{\mathrm{s}}
%\nc{\x}{\mathrm{x}}
%\nc{\y}{\mathrm{y}}
%\nc{\w}{\mathrm{w}}
%\nc{\z}{\mathrm{z}}
%\rnc{\r}{\mathrm{r}}
%\rnc{\k}{\mathrm{k}}
%\rnc{\j}{\mathrm{j}}
%\nc{\m}{\mathrm{m}}
%\nc{\n}{\mathrm{n}}
%\rnc{\i}{\mathrm{i}}
%\nc{\p}{\mathrm{p}}


\nc{\abbr}[1]{{\sc\lowercase{#1}}}

\rnc{\leq}{\leqslant}
\rnc{\geq}{\geqslant}
\rnc{\d}{\mathrm{d}}
\newenvironment{nouppercase}{%
  \let\uppercase\relax%
  \renewcommand{\uppercasenonmath}[1]{}}{}
\pagestyle{plain}




\title{\Large Math 154: Probability Theory, HW 7\vspace{-0.1cm}}





\usepackage{setspace}
\begin{document}
%\pdfrender{StrokeColor=gray,TextRenderingMode=2,LineWidth=0.01pt}
\setstretch{0.97}
\begin{nouppercase}
\maketitle
\end{nouppercase}
%%%
\section*{Due March 19, 2024 by 9am}
%%%
\emph{Remember, if you are stuck, take a look at the lemmas/theorems/examples from class, and see if anything looks familiar.}
%%%
\section{Getting to know the central limit theorem}
%%%
%%%
\subsection{Approximating a complicated expectation}
%%%
Let $\{X_{i}\}_{i=1}^{\infty}$ be i.i.d. random variables such that $\mathbb{P}[X_{i}=\pm1]=\frac12$. 
%%%
\begin{enumerate}
\item Show that $\E X_{i}=0$ and $\mathrm{Var}(X_{i})=1$ for all $i$.
\item Define $Y_{N}:=N^{-1/2}\sum_{i=1}^{N}X_{i}$. Using the central limit theorem, show
%
\begin{align*}
\lim_{N\to\infty}\E|Y_{N}|=\int_{\R}|x|\frac{1}{\sqrt{2\pi}}e^{-\frac{x^{2}}{2}}\d x.
\end{align*}
%
\item Compute $\lim_{N\to\infty}\E|Y_{N}|$ by evaluating the integral in part (2).
\end{enumerate}
%%%
%%%
\subsection{Approximating a complicated sum}
%%%
Fix any $x\geq0$. 
%%%
\begin{enumerate}
\item Explain why for any $k\geq0$, we have $2^{-N}\binom{N}{k}=\mathbb{P}[S_{N}=k]$, where $S_{N}\sim\mathrm{Bin}(N,\frac12)$ is a sum of $N$ independent $\mathrm{Bern}(\frac12)$.
\item Show that $2S_{N}-N$ is a sum of $N$ i.i.d. random variables with mean zero and variance $1$. Also show that 
%
\begin{align*}
\sum_{\substack{k:N^{-1/2}|2k-N|\leq x}}2^{-N}\binom{N}{k}&=\mathbb{P}\left(-x\leq \frac{2S_{N}-N}{N^{1/2}}\leq x\right)
\end{align*}
%
\item Show that as $N\to\infty$, we have 
%
\begin{align*}
\sum_{\substack{k:|2k-N|\leq xN^{1/2}}}2^{-N}\binom{N}{k}\to\int_{-x}^{x}\frac{1}{\sqrt{2\pi}}e^{-\frac{u^{2}}{2}}\d u.
\end{align*}
%
\item (Bonus, +2pt; please do not ask the CAs for help on this one): Show that 
%
\begin{align*}
\sum_{\substack{k:\\N^{-1/2}|k-N|\leq x}}\frac{N^{k}}{k!}e^{-N}\to_{N\to\infty}\int_{-x}^{x}\frac{1}{\sqrt{2\pi}}e^{-\frac{u^{2}}{2}}\d u.
\end{align*}
%
(\emph{Hint}: its the same argument; your job is to figure out exactly why.)
\end{enumerate}
%%%
%%%
\subsection{Stein's method}
%%%
We showed before that if $Z\sim N(0,1)$, then for any smooth function $f:\R\to\R$, we have $\E f'(Z)=\E Zf(Z)$. Conversely, suppose $W$ satisfies the property that for all smooth functions $f$, we have $\E f'(W)=\E Wf(W)$.
%%%
\begin{enumerate}
\item Show that $\E W=0$ and $\E W^{2}=1$ and $\E W^{3}=0$ and $\E W^{4}=3$.
\item (Bonus, +2pt; please do not ask the CAs for help on this one): Show that $W\sim N(0,1)$.
\end{enumerate}
%%%
Note that this gives a new way of proving the central limit theorem. There are interpretations of this method from physics (in fact, the physicists may argue this is the \emph{right} way to prove the CLT); please see me if you would like to discuss this.
%%%
\subsection{A little exercise about Fourier transforms}
%%%
Suppose $X_{N}\to X$ and $Y_{N}\to Y$ in distribution. 
%%%
\begin{enumerate}
\item Suppose also that $X_{N},Y_{N}$ are independent for each $N$, and that $X,Y$ are independent. Show that $X_{N}+Y_{N}\to X+Y$. (\emph{Hint}: use the Levy continuity theorem)
\item Give a counterexample to the above when we remove the independence assumptions.
\end{enumerate}
%%%
%%%
\subsection{The moment method}
%%%
Let $\{X_{i}\}_{i=1}^{\infty}$ be i.i.d. random variables such that $\E X_{i}=0$ and $\E X_{i}^{2}=1$ and $\E |X_{i}|^{3}<\infty$ for all $i$. Define $S_{N}=N^{-1/2}(X_{1}+\ldots+X_{N})$.
%%%
\begin{enumerate}
\item By expanding, show that 
%
\begin{align*}
\E S_{N}^{3}=N^{-\frac32}\sum_{i=1}^{N}\E X_{i}^{3}+N^{-\frac32}\sum_{1\leq i\neq j\leq N}3\E X_{i}^{2}\E X_{j}+N^{-\frac32}\sum_{i\neq j, j\neq k, i\neq k}\E X_{i}X_{j}X_{k}.
\end{align*}
%
\item Show that $\E S_{N}^{3}\to0$ as $N\to\infty$.
\item (Bonus, +1pt; please do not ask the CAs for help on this one): Assume now that $\E|X_{i}|^{4}<\infty$ for all $i$. Show that $\E S_{N}^{4}\to3$ by the same type of expansion argument.
\end{enumerate}
%%%






\end{document}
%%%